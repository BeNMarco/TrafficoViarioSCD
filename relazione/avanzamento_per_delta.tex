\section{Strategia di avanzamento delle entità di tipo veicolo, bici o pedone}
Un primo approccio per l'avanzamento delle entità passive (veicoli, bici,
pedoni) era quello di suddividere la strada in segmenti di breve lunghezza e
permettere l'attraversamento di uno spezzone di strada ad una sola entità per
volta.
Questa strategia porta ad un avanzamento per distanze di sicurezza, ma
l'aggiornamento della nuova posizione avrebbe portato ad un avanzamento ``a
scatti'' del sistema, al di la di eventuali ottimizzazioni.

Quello che in realtà occorre considerare nell'avanzamento delle entità sono le
proprietà intrinseche del moto di avanzamento, ovvero il tempo, lo spazio,
l'accelerazione e la decelerazione. La strategia scelta segue una logica di
avanzamento secondo il modello \ac{IDM}~\cite{treiber2000microscopic}.
Seguendo il modello \ac{IDM}, l'avanzamento avviene considerando il
\textbf{\textit{delta}} di tempo nel quale il mezzo deve avanzare e le proprietà
del veicolo stesso. In relazione a questi parametri viene calcolata una
percentuale sull'accelerazione massima con cui il mezzo è stato configurato.
Infine, utilizzando il valore dell'accelerazione ottenuto, viene aggiornata la
velocità corrente e la posizione finale del mezzo.
Questi nuovi valori saranno validi alla fine del delta di tempo relativo al
periodo in cui sono stati calcolati. Di seguito vengono riportate le funzioni
matematiche utilizzate per il calcolo dello spostamento delle entità:

\begin{equation}
s^{*}(t)=s(0)+Tv(t)+\frac{v(t)\Delta{v(t)}}{2\sqrt{ab}}
\end{equation}
\begin{equation}
a(t)=a[1-(\frac{v(t)}{v_{0}})^4-(\frac{s^{*}(t)}{s(t)})^2]
\end{equation}
dove
\begin{align*}
~a =&~\text{massima accelerazione possibile}\\
v_{0} =&~ \text{elocità desiderata} \\
v(t) =&~ \text{velocità corrente} \\
s(t) =&~ \text{distanza corrente dal mezzo che sta davanti} \\
s_{0} =&~ \text{quantità minima di avanzamento} \\
s^*(t) =&~ \text{distanza calcolata in funzione dei parametri di configurazione}
\\ &~ \text{dei mezzi interessati} \\
T =&~ \text{parametro di controllo per regolamentare la velocità} \\
b =&~ \text{decelerazione massima}
\end{align*}

% $a$ è la massima accelerazione possibile; $v_{0}$ è la velocità desiderata;
% $v(t)$ è la velocità corrente; $s(t)$ è la distanza corrente dal mezzo che sta
% davanti; $s_{0}$ è la quantità minima di avanzamento; $s^*(t)$ è la distanza
% calcolata in funzione dei parametri di configurazione dei mezzi interessati; $T$
% è un altro parametro di controllo per regolamentare la velocità; $b$ è la
% decelerazione massima.

Infine è possibile aggiornare la velocità corrente della macchina e lo step di
avanzamento($ns(t)$): 

\begin{equation}
v(t)= v(t)+a\Delta
\end{equation}
\begin{equation}
ns(t)= v(t)\Delta+0.5a\Delta^2
\end{equation}

con $\Delta$ uguale al tempo desiderato per calcolare la posizione del mezzo
alla fine del $\Delta$ stesso. 

Il modello permette quindi di rappresentare una realtà continua relativa
all'avanzamento delle entità, discretizzando il tempo per una quantità $\Delta$.
In pratica, viene dato in input al modello \ac{IDM} lo stato del mezzo,
contenente la posizione corrente e i parametri di avanzamento.
Il modello ritornerà dei valori relativi all'aggiornamento della posizione delle
entità, i quali saranno validi per il modello a realtà continua alla fine
del $\Delta$ di tempo relativo al periodo in cui sono stati calcolati.

Il modello presentato richiede quindi una configurazione di alcuni parametri,
tra i quali le proprietà di moto delle entità e un parametro di sistema, ovvero
il $\Delta$. La configurazione dei parametri delle entità può essere lasciata
all'utente, predisponendo dei valori di default al fine di accelerare il
processo di configurazione delle entità.

Si è convenuto di assegnare a $\Delta$ un valore costante, quindi non
configurabile dall'utente. La scelta di tale quantità, ovvero la durata
del quanto di discretizzazione deve seguire la seguente logica:
se il $\Delta$ viene tarato con valori grandi (nell'ordine dei secondi) allora
il sistema eseguirebbe meno calcoli per completare gli spostamenti delle entità.
D'altro canto, al crescere del $\Delta$ la simulazione risulterebbe rallentata
dal punto di vista dell'avvenimento di alcuni eventi dovuti per conseguenza di
altri, ovvero le entità sarebbero poco reattive e ritarderebbero le loro azioni
in funzione della grandezza del $\Delta$. Se il $\Delta$ è troppo piccolo
(nell'ordine dei millisecondi) il sistema si troverebbe nella situazione di
eseguire molti calcoli per completare lo spostamento, le entità sarebbero
reattive, ma la quantità di avanzamento effettiva di una entità sarebbe minuta e
inutile al fine della reattività instantanea delle altre entità presenti nel
sistema. Il valore del $\Delta$ da noi scelto è \textbf{\textit{0.5 secondi}},
cosi da permettere il giusto compresso tra step di avanzamento delle entità e
reattività delle entità.