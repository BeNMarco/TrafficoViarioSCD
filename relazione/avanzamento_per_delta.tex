\section{Strategia di avanzamento delle entità}
Un primo approccio per l'avanzamento era quello di suddividere la strada in segmenti di breve lunghezza e permettere l'attraversamento di un segmento una sola entità per volta; questa strategia porta ad un avanzamento per distanze di sicurezza e l'aggiornamento della nuova posizione avrebbe portato ad un avanzamento "a scatti" del sistema al di la di eventuali ottimizzazioni. \\
Quello che in realtà occorre considerare nell'avanzamento delle entità sono le proprietà intinseche dell'avanzamento, ovvero il tempo, lo spazio, l'accelerazione  e la decelarazione. La strategia scelta segue una logica di avanzamento secondo il modello IDM (Intelligent-Driver Model).
L'avanzamento quindi avviene considerando, il \textbf{\textit{delta}} di tempo nel quale il mezzo deve avanzare, le proprietà del mezzo interessato, quali velocità corrente, velocità massima, coefficiente di accelerazione e di decelarazione, \textbf{\textit{distanza}} dal mezzo che sta davanti; in relazione a questi parametri viene calcolata una percentuale sull'accelerazione con cui il mezzo è stato configurato. Infine viene aggiornata la velocità corrente e la posizione finale del mezzo alla fine del delta di tempo imposto.

\begin{equation}
s^{*}(t)=s(0)+Tv(t)+\frac{v(t)\Delta{v(t)}}{2\sqrt{ab}}
\end{equation}
\begin{equation}
a(t)=a[1-(\frac{v(t)}{v_{0}})^4-(\frac{s^{*}(t)}{s(t)})^2]
\end{equation}

$a$ è la massima accelerazione possibile; $v_{0}$ è la velocità desiderata; $v(t)$ è la velocità corrente; $s(t)$ è la distanza corrente dal mezzo che sta davanti; $s_{0}$ è la quantità minima di avanzamento; $s^*(t)$ è la distanza calcolata in funzione dei parametri di configurazione dei mezzi interessati; $T$ è un altro parametro di controllo per regolamentare la velocità.\\
Infine è possibile aggiornare la velocità corrente della macchina e lo step di avanzamento($ns(t)$): \\

\begin{equation}
v(t)= v(t)+a\Delta
\end{equation}
\begin{equation}
ns(t)= v(t)\Delta+0.5a\Delta^2
\end{equation}

con $\Delta$ uguale al tempo desiderato per calcolare la posizione del mezzo alla fine del $\Delta$ stesso. \\
Il modello presentato permette quindi di rappresentare una realtà continua relativa all'avanzamento delle entità, discretizzando il tempo per una quantità $\Delta$; in pratica viene dato in input al modello IDM lo stato del mezzo contenente la posizione corrente e i parametri di avanzamento; il modello ritornerà l'aggiornamento della posizione del mezzo alla fine del $\Delta$.