\section{Analisi del dominio del problema}
In questa sezione viene descritta la consegna data integrata con i nostri requisiti funzionali. La consegna iniziale prevedeva la realizzazione di una simulazione del traffico in una città. \\
L'obiettivo dell'analisi del dominio è quello di fissare dall'inizio quello che il simulatore deve fare in termini di funzionalità di cui l'attore dello spostamento deve essere dotato.

\subsection{Entità rappresentate nella mappa della città}
\label{firstmappa}
La mappa della città è caratterizzata da strade dotate di doppia corsia per senso di marcia, d'ora in poi \textit{\textbf{strade principali}}, marciapiedi, piste ciclabili, \textit{\textbf{strade di ingresso}} al traffico, incroci, dotati di semaforo, con 3 o 4 strade e attraversamenti pedonali.\\
Sono stati identificati come attori del traffico della città simulata: veicoli (auto e autobus), biciclette e pedoni.

\subsection{Requisiti sullo spostamento delle entità}
\begin{enumerate}
\item {\textit{requisito relativo al mezzo di spostamento degli attori}}: deve essere data la possibilità di configurare il mezzo di spostamento dell'attore che verrà adottato poi per tutta la durata della simulazione;
\item {\textit{requisito relativo ai luoghi di destinazione degli attori}}: un attore deve poter essere configurato in modo tale da muoversi nella città al fine di spostarsi verso un certo luogo di destinazione.
Un attore quindi una volta raggiuta la destinazione dovrà reimmetersi nel traffico per raggiungere il luogo da cui era partito, iterando quindi l'avanzamento da luogo di partenza a luogo di destinazione e viceversa. L'obiettivo di questo requisito è quello di avere una situazione in cui nella città vi sia sempre del traffico nel caso in cui almeno un'entità sia presente.
\end{enumerate}

\subsection{Convenzioni sulla mappa}
A seguito dell'analisi è stato convenuto riportare delle regole sulla realizzazione della mappa avendo fissato a priori le entità sulle quali gli attori dovevano eseguire lo spostamento, vedi \ref{firstmappa}.
\begin{enumerate}
\item una strada principale deve essere delimitata almeno da un incrocio e al più da due incroci;
\item una strada di ingresso è una strada adibita all'entrata o all'uscita da o verso un certo luogo;
\item una strada principale presenta due lati ai quali possono essere inserite delle strade di ingresso; quindi se su una stessa strada principale vengono inserite più strade di ingresso è convenuto imporre che queste strade tengano una distanza minima l'una dall'altra, indipendentemente dal lato della strada principale in cui vengono inserire; l'obiettivo è quello di evitare che si formi un incrocio tra strade di ingresso non regolamentate da semaforo;
\item ogni strada principale contiene esattamente una fermata per l'autobus posta a seconda della configurazione della mappa su uno dei due lati della strada principale stessa.
\end{enumerate}

\subsection{Altri requisiti rilevati dall'analisi del dominio}
Nell'analisi del dominio del problema sono state identificate quindi le entità che sono rilevanti per la realizzazione del simulatore e sono state definite delle regole e quindi delle limitazioni funzionali per la realizzazione della mappa. \\
Sono emerse inoltre le seguenti funzionalità e requisiti a seguito di un'analisi a livello anche progettuale:
\begin{enumerate}
\item un entità del tipo veicolo che deve svoltare a sinistra al prossimo incrocio dovrà immettersi ad un certo punto nella corsia più a sinistra della strada che sta percorrendo; se il veicolo dovrà andare a destra dovrà immettersi nella corsia più a destra della strada che sta percorrendo e indifferentemente su entrambe le corsie se vuole procedere in direzione dritto;
\item un entità del tipo veicolo deve poter agevolmente cambiare corsia su una strada principale in modo tale da poter raggiungere la corsia interessata in relazione alla destinazione da perseguire;
\item il percorso che l'entità di tipo veicolo, bipede, o bici deve percorrere deve essere fissato a priori. \\
Questa è una limitazione per la simulazione, ma per semplicità anche di implementazione è stato convenuto definire un percorso statico piuttosto che dinamico;
\item un pedone o una bici non eseguono dei sorpassi, ma procedono secondo una politica FIFO nell'avanzamento rispettivamente nel marciapiede e nella pista ciclabile. 
\end{enumerate}

\newpage

