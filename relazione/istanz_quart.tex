\section{Protocollo per l'istanziazione di una partizione di tipo quartiere}
La partizione può iniziare ad operare solo dopo che le componenti di name server e server gps sono state correttamente istanziate; la componente dovrà poi eseguire le seguenti operazioni:
\begin{enumerate}
\item deve essere configurata la mappa del quartiere, in relazione al file di configurazione dato in input alla componente, e quindi occorre creare tutte le entità attive che la partizione vede partecipe (strade, incroci), tutte le entità passive (veicoli, bici e pedoni), e tutte quelle risorse che possono essere riferite da una partizione remota (gestore delle risorse di tipo autobus, gestore del servizio di locazione delle entità passive, risorsa utilizzata per ottenere informazioni sul quartiere, risorse in gestione alle entità attive per memorizzare lo stato di avanzamento delle entità passive);
\item il quartiere deve poi procedere a registrare al name server le risorse riferibili da remoto; tale operazione deve essere un'operazione atomica che può o meno andare a buon fine, per esempio essa sarà rifiutata se un quartiere con lo stesso identificativo è già stato registrato;
\item se l'operazione precedente è avvenuta con successo, la componente del quartiere dovrà procedere con la registrazione della configurazione della mappa al server gps e quindi a operazione terminata dovrà comunicare al name server che il quartiere in questione ha correttamente configurato la propria mappa nel server gps; a questo punto il quartiere sarà un quartiere correttamente istanziato ed ogni altro quartiere avrà la visibilità del quartiere appena creato; fintanto che il quartiere non ha registrato la sua presenza al name server, eventuali operazioni di errore nella partizione non sono compromettenti per la consistenza dello stato dell'intero sistema, mentre se il quartiere ha eseguito la propria registrazione nel name server allora ogni errore che avverrà durante l'esecuzione di tale partizione genera uno stato inconsistente per il sistema e sarà necessario quindi eseguire un'operazione di riavvio; 
\item se la partizione si trova in stato di errore dovrà terminare, altrimenti può procedere con l'esecuzione del protocollo di sincronizzazione, vedi \ref{protosynch}.
\end{enumerate}