\section{Utilizzo del prototipo}
Il prototipo viene distribuito già compilato e pronto all'utilizzo all'interno
di una macchina virtuale, realizzata tramite VirtualBox.

Il file che viene consegnato è un disco virtuale contenente un'installazione di
Ubuntu Server 14.04. La versione di GNAT utilizzata per lo sviluppo del progetto
è la GPL 2013 64bit.

La macchina virtuale utilizzata per l'installazione dei componenti è stata
configurata con 2048 MB di memoria RAM, 2 processori, virtualizzazione abilitata
e due interfacce di rete. La prima è configurata come NAT (utilizzata per
l'accesso a internet) mentre la seconda come ``Only-Host network''. La seconda
interfaccia permette di connettersi via ssh alla macchina virtuale e di
visualizzare la simulazione sul proprio browser\footnote{Lo sviluppo è stato
interamente svolto utilizzando il browser Google Chrome.}. 

Una volta avviata la macchina virtuale è necessario effettuare il login. Le
credenziali sono:
\begin{description}
	\item[username:] scd
	\item[password:] scd
\end{description}

\subsection{Compilazione}
Tutto il progetto e i sorgenti sono contenuti nella cartella
\texttt{/home/scd/TrafficoViarioSCD/}.

Una volta entrati in questa cartella è possibile ricompilare il progetto
utilizzando il seguente comando:

\begin{center}
\texttt{po\_gnatdis -Pdefault default.cfg}
\end{center}

Questa fase non è necessaria perché il disco virtuale contiene già la
compilazione del progetto alla sua ultima versione.

\subsection{Fasi preliminari}
Tutti i file necessari per l'esecuzione del progetto sono contenuti nella
sottocartella \texttt{per\_eseguire}. Qualora si dovesse ricompilare il
progetto, è necessario eseguire lo script \texttt{copia\_eseguibili.sh} contenuto in
questa cartella.

Per testare il progetto si mettono a disposizione dell'utente 3 configurazioni.
Le configurazioni prevedono tutte l'utilizzo della stessa mappa ma il numero di
entità che popola la simulazione è diverso.

Per selezionare e preparare la configurazione è possibile utilizzare lo script
\texttt{helper.py} contenuto nella cartella principale del progetto.
Utilizzare il comando \texttt{config} per configurare il sistema. Sono presenti
tre configurazioni, rispettivamente identificate con 1, 2 e 3. Per selezionare
la configurazione 1, ad esempio, usare il comando

\begin{center}
	\texttt{./helper.py config -c 1}
\end{center}

\subsection{Avvio}
Per avviare la simulazione è possibile utilizzare lo script \texttt{helper.py}
ed eseguire le varie componenti, tramite il comando \texttt{start}.

Per ogni terminale è possibile eseguire più di un componente. Utilizzare i
seguenti parametri per avviare i vari componenti:
\begin{description}
	\item[-ns] avviare il name server
	\item[-cs] avviare il server centrale
	\item[-ws] avviare il web server
	\item[-q] avviare un quartiere. Assieme a questo parametro è necessario
	specificare che quartiere (uno o più) avviare. I quartieri disponibili sono 1,
	2 e 3
\end{description}

Se, per esempio, si vuole avviare la simulazione in due terminali, è possibile
usare i seguenti comandi:
\begin{center}
	\texttt{./helper.py start -ns -cs -q 1}
	
	\texttt{./helper.py start -ws -q 2 3}
\end{center}

Il primo comando avvia il name server, il server centrale e il quartiere 1
mentre il secondo avvia il web server e i quartieri 2 e 3.

\subsection{Visualizzazione}
Per visualizzare la simulazione utilizzare un browser e utilizzare l'indirizzo
IP della macchina virtuale, assegnato all'interfaccia di rete di tipo
``Host-Only''. Di default il web server viene avviato sulla porta 8080 ma è
possibile impostare un reindirizzamento delle porte, per usare la porta 80,
eseguendo lo script \texttt{portForsard.sh}. La password da utilizzare in caso
di richiesta è ``scd''.

La mappa principale visualizza le partizioni attive e le arre della mappa da
esse gestite. Le aree vengono disegnate appena una partizione si attiva.

Per visualizzare la simulazione dettagliata di un singolo quartiere posizionare
il puntatore sopra l'area desiderata, per evidenziarla, e poi cliccare.

\subsection{Terminazione}
Il sistema può essere terminato in due modi:
\begin{itemize}
	\item premendo il tasto ``Q'' nel terminale nel quale si è avviato, da solo o
	con altri componenti, il name server
	\item premendo il tasto di richiesta di terminazione, in una delle pagine di
	visualizzazione della simulazione
\end{itemize}
