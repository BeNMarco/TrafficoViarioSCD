\section{Gestione semafori}
I semafori vengono utilizzati per regolamentare l'attraversamento delle entità
negli incroci; dato che il tempo è discretizzato quello che conviene fare è
cambiare il colore del semaforo sulla base di un contatore del numero di quanti
trascorsi con il semaforo configurato con un determinato colore. La cosa
evidente è quella di far in modo che se in un certo quanto di tempo il semaforo
presenta un certo colore, tale colore sarà lo stesso per tutto il quanto e
potrà cambiare al più dal quanto di sistema successivo. Un incrocio può avere
delle strade che appartengono a quartieri diversi dal quartiere a cui
l'incrocio appartiene; un requisito di sistema deve essere che se una strada
principale richieda in un certo quanto il colore del semaforo, tale colore
rimarrà tale per ogni altra richiesta avvenuta nel quanto di sistema stesso.
Sfruttando il protocollo di sincronizzazione è possibile trovare il momento
giusto in cui è possibile cambiare il colore dei semafori. Infatti quando il
sistema è sincronizzato allora tutte le entità attive hanno svolto il loro
lavoro in relazione al quanto di tempo precedente; quindi quando un quartiere
riceverà la notifica che tutti gli altri quartieri sono pronti per una nuova
sincronizzazione allora in quel momento può avvenire la gestione dei semafori
permettendo cosi che per un task di un certo quartiere, ad operazione di
sincronizzazione eseguita tutti i semafori degli incroci sono stati gestiti e
se necessario avranno quindi cambiato il loro colore.\\
La politica scelta per aggiornare lo stato dei semafori, si basa sulla
considerazione che se un semaforo di una certa strada è verde, allora sarà
verde anche quello della strada opposta, ovvero nella stessa direzione; mentre
le strade perpendicolari avranno colore del semaforo rosso. Infine gli
attraversamenti pedonali degli incroci sono gestiti in modo tale che a semaforo
verde per le entità di tipo bici/pedoni tutti gli attraversamenti pedonali
dell'incrocio avranno semaforo verde. \\
Un semaforo resterà verde per un certo numero di quanti fissato a priori
(variabile a seconda della tipologia dell'entità passiva), e la gestione degli
attraversamenti deve avvenire secondo la seguente strategia:
\begin{enumerate}
\item si scelgano quali strade devono avere per prime il semaforo verde, si 
abilitano tali semafori (favorendo quindi l'avanzamento dei veicoli) e si
disabilitano i semafori di bici/pedoni;
\item si controlla il numero di quanti trascorsi, per i quali il semaforo è
rimasto verde;
\item se il numero di quanti entro i quali il semaforo doveva essere verde ha
raggiunto il numero massimo, allora se il semaforo abilitato è il semaforo dei
veicoli, lo si deve disabilitare e abilitare quello di bici/pedoni; se il
semaforo abilitato era quello per le bici e pedoni allora lo si deve
disabilitare e abilitare quello dei veicoli con colore di semaforo rosso per
quelle strade che per ultime avevano il semaforo verde, dando il colore verde
ai semafori di quelle strade quindi che per ultime avevano il colore rosso.\\
Ritorna in \textit{2)}.
\end{enumerate}
 