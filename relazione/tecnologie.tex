\section{Tecnologie utilizzate}
Le tecnologie utilizzate per lo sviluppo del prototipo sono state scelte
valutando principalmente il modello di concorrenza offerto dal linguaggio.
Quindi è stato scelto Ada dato che il modello di concorrenza che il linguaggio
offre si presta in modo tale da distinguere in modo chiaro le entità
protagoniste in una realtà concorrente: task, risorse, code di attesa, ecc. Per
quanto riguarda la distribuzione è stato sfruttato sempre il modello di
distribuzione di Ada, ovvero Ada \ac{DSA}, e quindi il middleware di
PolyORB~\cite{polyorb}; la scelta di utilizzare DSA limita la portabilità
dell'applicazione, ma è stato scelto comunque a discapito della portabilità,
valutando principalmente la trasparenza con cui DSA permette di realizzare un
sistema distribuito, e quindi la quantità di lavoro richiesto per implementare
il prototipo.

Per l'implementazione del web server è stato utilizzato il framework
\ac{AWS}~\cite{aws}. \ac{AWS} permette di realizzare web application complete e
ricche, utilizzando il linguaggio di programmazione Ada. La scelta di questo
framework è stata dettata dalla facilità di utilizzo e di integrazione con il
resto del sistema. La possibilità di utilizzare Ada \ac{DSA} ha semplificato
notevolmente l'implementazione della comunicazione tra il web server e il resto
del sistema. 

La visualizzazione dello stato è stata realizzata utilizzando tecnologie di
sviluppo web. In particolare, è stata utilizzata la libreria JavaScript
Paper.js. Questo software ha permesso di disegnare la mappa e le entità che si
muovono su di essa in modo semplice e intuitivo, adattandosi pienamente al
modello di simulazione a traiettorie utilizzato dal sistema. La continua
ricezione di aggiornamenti di stato da parte dei client è stata resa possibile
grazie all'utilizzo di WebSocket. La scelta di utilizzare queste tecnologie è
stata motivata dalla facilità di sviluppo delle applicazione e dalla semplicità
con cui vengono distribuite. L'utente che vuole visualizzare la simulazione del
sistema deve solamente essere dotato dell'ultima versione del proprio browser
web\footnote{Le prestazioni migliori con il nostro sistema sono state osservate
utilizzando il web browser Chrome}. 
