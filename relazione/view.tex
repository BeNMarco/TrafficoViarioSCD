\section{Visualizzazione della simulazione}
La visualizzazione della simulazione è stata realizzata utilizzando un Web
Server, appositamente realizzato per lo scopo, e tecnologie web per la
realizzazione di contenuti web quali HTML5, JavaScript e WebSocket.

\subsection{Requisiti}\label{subsec:requisiti}
L'interfaccia grafica è stata progettata e realizzanta in modo tale da
soddisfare una serie di requisiti che derivano dalla natura distribuita del
sistema. 

I requisiti identificati per questo modulo sono riportati di seguiito.

\begin{description}
	\item[Scalabilità:] la rappresentazione grafica della simulazione deve 
	adattarsi alla dimensione del sistema. Deve essere garantita la possibilità 
	di visualizzare nel dettaglio lo stato di ogni singolo quartiere simulato dal
	sistema.

	\item[Coerenza:] la rappresentazione attuale della simulazione deve essere
	coerente con lo stato attuale del sistema. Nel dettaglio, la rappresentazione
	della posizione degli abitanti dei quartieri deve essere fedele con la
	posizione effettiva nella simulazione. Questa condizione viene rilassata 
	
	\item[Utilità:] oltre a visualizzare la simulazione, l'interfaccia grafica deve
	essere utilizzabile per la verifica e il controllo delle mappe in fase di
	sviluppo. Data la descrizione degli elementi presenti in un quartiere, la
	libreria di visualizzazione deve essere in grado di fornire tutte le misure
	necessarie al sistema per eseguire la simulazione.
	
	\item[]
	
\end{description}

\subsection{Requisiti di sistema}
Per 